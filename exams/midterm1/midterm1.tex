\documentclass[12pt]{article}

\include{preamble}

\title{Math 341 / 650 Spring 2020 \\ Midterm Examination One}
\author{Professor Adam Kapelner}

\date{Thursday, February 27, 2020}

\begin{document}
\maketitle

\noindent Full Name \line(1,0){410}

\thispagestyle{empty}

\section*{Code of Academic Integrity}

\footnotesize
Since the college is an academic community, its fundamental purpose is the pursuit of knowledge. Essential to the success of this educational mission is a commitment to the principles of academic integrity. Every member of the college community is responsible for upholding the highest standards of honesty at all times. Students, as members of the community, are also responsible for adhering to the principles and spirit of the following Code of Academic Integrity.

Activities that have the effect or intention of interfering with education, pursuit of knowledge, or fair evaluation of a student's performance are prohibited. Examples of such activities include but are not limited to the following definitions:

\paragraph{Cheating} Using or attempting to use unauthorized assistance, material, or study aids in examinations or other academic work or preventing, or attempting to prevent, another from using authorized assistance, material, or study aids. Example: using an unauthorized cheat sheet in a quiz or exam, altering a graded exam and resubmitting it for a better grade, etc.
\\

\noindent I acknowledge and agree to uphold this Code of Academic Integrity. \\

\begin{center}
\line(1,0){250} ~~~ \line(1,0){100}\\
~~~~~~~~~~~~~~~~~~~~~signature~~~~~~~~~~~~~~~~~~~~~~~~~~~~~~~~~~~~~~~~~~~~~ date
\end{center}

\normalsize

\section*{Instructions}

This exam is seventy five minutes and closed-book. You are allowed \textbf{one} page (front and back) of a \qu{cheat sheet.} You may use a graphing calculator of your choice. Please read the questions carefully. If the question reads \qu{compute,} this means the solution will be a number otherwise you can leave the answer in \textit{any} widely accepted mathematical notation which could be resolved to an exact or approximate number with the use of a computer. I advise you to skip problems marked \qu{[Extra Credit]} until you have finished the other questions on the exam, then loop back. I also advise you to use pencil. The exam is 100 points total plus extra credit. Partial credit will be granted for incomplete answers on most of the questions. \fbox{Box} in your final answers. NO FOOD but drinks okay. Good luck!

\pagebreak


\begin{table}[htp]
\centering
\small
\begin{tabular}{l | llll}
Distribution                  & Quantile  & PMF / PDF  &CDF       & Sampling  \\ 
of r.v. &  Function & function         & function &  Function \\ \hline
beta & \texttt{qbeta}($p$, $\alpha$, $\beta$)             
& \texttt{d-}($x$, $\alpha$, $\beta$)
& \texttt{p-}($x$, $\alpha$, $\beta$) 
& \texttt{r-}($\alpha$, $\beta$) \\
betabinomial & \texttt{qbetabinom}($p$, $n$, $\alpha$, $\beta$)              
& \texttt{d-}($x$, $n$, $\alpha$, $\beta$)
& \texttt{p-}($x$, $n$, $\alpha$, $\beta$) 
& \texttt{r-}($n$, $\alpha$, $\beta$) \\

betanegativebinomial & \texttt{qbeta\_nbinom}($p$, $r$, $\alpha$, $\beta$) 
& \texttt{d-}($x$, $r$, $\alpha$, $\beta$)
& \texttt{p-}($x$, $r$, $\alpha$, $\beta$) 
& \texttt{r-}($r$, $\alpha$, $\beta$) \\

binomial & \texttt{qbinom}($p$, $n$, $\theta$) 
& \texttt{d-}($x$, $n$, $\theta$)
& \texttt{p-}($x$, $n$, $\theta$) 
& \texttt{r-}($n$, $\theta$) \\

exponential & \texttt{qexp}($p$, $\theta$) 
& \texttt{d-}($x$, $\theta$) 
& \texttt{p-}($x$, $\theta$) 
& \texttt{r-}($\theta$) \\

gamma & \texttt{qgamma}($p$, $\alpha$, $\beta$) 
& \texttt{d-}($x$, $\alpha$, $\beta$)
& \texttt{p-}($x$, $\alpha$, $\beta$) 
& \texttt{r-}($\alpha$, $\beta$) \\

geometric & \texttt{qgeom}($p$, $\theta$) 
& \texttt{d-}($x$, $\theta$)
& \texttt{p-}($x$, $\theta$) 
& \texttt{r-}($\theta$) \\

inversegamma & \texttt{qinvgamma}($p$, $\alpha$, $\beta$) 
& \texttt{d-}($x$, $\alpha$, $\beta$)
& \texttt{p-}($x$, $\alpha$, $\beta$) 
& \texttt{r-}($\alpha$, $\beta$) \\

negative-binomial & \texttt{qnbinom}($p$, $r$, $\theta$) 
& \texttt{d-}($x$, $r$, $\theta$) 
& \texttt{p-}($x$, $r$, $\theta$) 
& \texttt{r-}($r$, $\theta$) \\

normal (univariate) & \texttt{qnorm}($p$, $\theta$, $\sigma$) 
& \texttt{d-}($x$, $\theta$, $\sigma$)
& \texttt{p-}($x$, $\theta$, $\sigma$) 
& \texttt{r-}($\theta$, $\sigma$) \\

%normal (multivariate) & 
%& \multicolumn{2}{l}{\texttt{dmvnorm}($\x$, $\muvec$, $\bSigma$)} 
%& \texttt{r-}($\muvec$, $\bSigma$) \\

poisson & \texttt{qpois}($p$, $\theta$) 
& \texttt{d-}($x$, $\theta$)
& \texttt{p-}($x$, $\theta$) 
& \texttt{r-}($\theta$) \\

T (standard) & \texttt{qt}($p$, $\nu$) 
& \texttt{d-}($x$, $\nu$) 
& \texttt{p-}($x$, $\nu$)
& \texttt{r-}($\nu$) \\

%T (nonstandard) & \texttt{qt.scaled}($p$, $\nu$, $\mu$, $\sigma$) 
%& \texttt{d-}($x$, $\nu$, $\mu$, $\sigma$)
%& \texttt{p-}($x$, $\nu$, $\mu$, $\sigma$) 
%& \texttt{r-}($\nu$, $\mu$, $\sigma$) \\

uniform & \texttt{qunif}($p$, $a$, $b$) 
& \texttt{d-}($x$, $a$, $b$)
& \texttt{p-}($x$, $a$, $b$) 
& \texttt{r-}($a$, $b$) \\


\end{tabular}
\caption{Functions from $\texttt{R}$ (in alphabetical order) that can be used on this exam with their arguments. The hyphen in colums 3, 4 and 5 is shorthand notation for the full text of the r.v. which can be found in column 2.
}
\label{tab:eqs}
\end{table}

\problem Let $\mathcal{F}$ be binomial with known sample size $n = 3$. The data is all \qu{successes} i.e. $x = 3$. For all questions that have numerical answers, use three significant digits e.g. 0.123 and 1.23$\times 10^{-5}$ or fractions.

\benum

\subquestionwithpoints{2} Find the maximum likelihood estimate for $\theta$. \spc{0}

\subquestionwithpoints{4} What is the main problem with your estimate in (a)? \spc{2}

\subquestionwithpoints{3} Find the $CI_{\theta, 99\%}$. \spc{0.5}

\subquestionwithpoints{4} Does the interval in (c) fulfill the second goal of statistical inference? Yes / no and explain your answer. \spc{6}


\subquestionwithpoints{2} We will now conduct Bayesian inference. Consider the reduced parameter space $\Theta_0 = \braces{0.50, 0.99} \subset \Theta = (0, 1)$. We believe strongly in $\theta = 0.5$ but we want to give some credence to the alternate theory. Thus we establish a prior of 

\beqn
\prob{\theta} = \begin{cases}
0.50 \withprob 0.9 \\ 
0.99 \withprob 0.1 \\ 
\end{cases}
\eeqn

Is this the \qu{prior of indifference} for the reduced parameter space? Yes / no and explain.\spc{2}

\subquestionwithpoints{5} Find $\thetahatmap$. \spc{3}

\subquestionwithpoints{5} Find $\prob{X = x}$. \spc{4}

\subquestionwithpoints{5} Find the posterior predictive probability $\cprob{X_* = 1}{X = x}$ where $X_*$ denotes the next observation. \spc{5}


\subquestionwithpoints{3} We will now consider the entire parameter space for the binomial model i.e. $\Theta = (0, 1)$. We will use the prior $\theta \sim \betanot{\half}{\half}$. We will see later in class that this is called the \qu{Jeffrey's Prior}. Is this an uninformative prior? Yes / no and explain.\spc{2}

\subquestionwithpoints{2} Is this the \qu{prior of indifference}? Yes / no and explain. \spc{1}

\subquestionwithpoints{2} How many pseudosuccesses and pseudofailures is within this prior?\spc{1}

\subquestionwithpoints{3} What is $\expe{\theta}$?\spc{0}

\subquestionwithpoints{5} Find $\cprob{\theta}{X = x}$. \spc{1}

\subquestionwithpoints{6} Draw $\cprob{\theta}{X = x}$ to the best of your ability. Label all axes and critical points. \spc{6}

\subquestionwithpoints{2} Does $\thetahatmap$ exist? Yes / No. \spc{-0.5}

\subquestionwithpoints{4} Find $\thetahatmmse$ and denote it in the illustration in (n). \spc{1}

\subquestionwithpoints{4} What is the proportion of shrinkage towards the prior expectation if you employ the posterior expectation as your point estimate? \spc{1}

\subquestionwithpoints{5} Find the $CR_{\theta, 99\%}$. \spc{1}
\subquestionwithpoints{5} Find the $HDR_{\theta, 99\%}$ and denote it in the illustration in (n). \spc{1}

\subquestionwithpoints{10} Test if $\theta > 0.5$. Write out the hypotheses and declare the $\alpha$ level you are comfortable with. Estimate the Bayesian $p$-value from the illustration of the posterior distribution in (n) and provide the conclusion of the test. \spc{8}

\subquestionwithpoints{4} Find the posterior predictive probability $\cprob{X_* = 1}{X = x}$ where $X_*$ denotes the next observation. \spc{1}

\subquestionwithpoints{3} What is your best guess of $X_*$? \spc{2}
\eenum

\problem Consider $\Xoneton \iid \betanot{1}{\theta}$.

\benum

\subquestionwithpoints{6} Find $\mathcal{L}\parens{\theta ; \Xoneton}$. Simplify so that your answer does not include the $B(\cdot,\cdot)$ function or the $\Gamma(\cdot)$ function.\spc{4}

\subquestionwithpoints{3} Find $\loglik{\theta ; \Xoneton}$. Simplify as much as possible.\spc{2}

\subquestionwithpoints{3} Find $\thetahatmle$. \spc{3}

\subquestionwithpoints{8} [Extra Credit] Consider $\Xoneton \iid \betanot{\theta_1}{\theta_2}$. Find the MLE for $\theta_1$ and the MLE for $\theta_2$. Partial credit is given.\spc{3}
\eenum


\end{document}
